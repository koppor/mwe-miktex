\documentclass[peerreview,a4paper,english]{IEEEtran}[2015/08/26]
\usepackage[ngerman,main=english]{babel}
\addto\extrasenglish{\languageshorthands{ngerman}\useshorthands{"}}
\usepackage[hyphens]{url}
\makeatletter
\g@addto@macro{\UrlBreaks}{\UrlOrds}
\makeatother
\usepackage[zerostyle=b,scaled=.75]{newtxtt}
\usepackage[T1]{fontenc}
\usepackage[
  babel=true, % Enable language-specific kerning. Take language-settings from the languge of the current document (see Section 6 of microtype.pdf)
  expansion=alltext,
  protrusion=alltext-nott, % Ensure that at listings, there is no change at the margin of the listing
  final % Always enable microtype, even if in draft mode. This helps finding bad boxes quickly.
        % In the standard configuration, this template is always in the final mode, so this option only makes a difference if "pros" use the draft mode
]{microtype}
\DisableLigatures{encoding = T1, family = tt* }
\usepackage{graphicx}
\usepackage{diagbox}
\usepackage{xcolor}
\usepackage{listings}
\definecolor{eclipseStrings}{RGB}{42,0.0,255}
\definecolor{eclipseKeywords}{RGB}{127,0,85}
\colorlet{numb}{magenta!60!black}
\usepackage[autostyle=true]{csquotes}
\defineshorthand{"`}{\openautoquote}
\defineshorthand{"'}{\closeautoquote}
\usepackage{booktabs}
\usepackage{paralist}
\usepackage{lipsum}
\usepackage[math]{blindtext}
\usepackage{mwe}
\usepackage[realmainfile]{currfile}
\usepackage{tcolorbox}
\hyphenation{op-tical net-works semi-conduc-tor}
\begin{document}
\title{Quick start for LaTeXing with IEEEtran.cls for\\ IEEE Computer Society Conferences}
\maketitle
\begin{abstract}
Test
\end{abstract}

\emph{Purpose and scope of your entire report}. The purpose of your entire report is to make a \emph{scientific argument using the scientific method}. A scientific argument always has the following steps that all must come in this order.
%
\begin{itemize}
    \item[SM1] \emph{Explicate the assumptions and state of the art} on which you are going to conduct your research to investigate your research problem/test the hypothesis.
    \item[SM2] Clearly and precisely \emph{formulate a research problem or hypothesis}.
    \item[SM3] \emph{Describe the (research) method} that you followed to investigate the problem / to test the hypothesis in a way that \emph{allows someone else to reproduce your steps}. The method must includes steps and criteria for evaluating whether you answered your question successfully or not.
    \item[SM4] \emph{Provide execution details} on how you followed the method in the given, specific situation.
    \item[SM5] \emph{Report your results} by describing and summarizing your measurements. You must not interpret your results.
    \item[SM6] \emph{Now interpret your results} by contextualizing the measurements and drawing conclusion that lead to answering your research problem or defining further follow-up research problems.
\end{itemize}
\end{document}
